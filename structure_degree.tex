
%polskie nagłówki
\usepackage[polish]{babel}

%kolory
\usepackage[x11names]{xcolor}

%zalącznie skekcji i pdfow
\usepackage{subfiles}
\usepackage{pdfpages}

%geometria stron
\usepackage[top=2cm,bottom=3cm,left=2cm,right=2cm]{geometry} % This package allows the editing of the page layout

%grafiki
\usepackage{graphicx}  % This package allows the importing of images
\graphicspath{{Figures/}} % Set the default folder for images
\usepackage{tikz}

%zalączanie incscapa
\usepackage{import}
\usepackage{xifthen}
\usepackage{transparent}
\newcommand{\incfig}[1]{\def\svgwidth{\columnwidth}\input{#1.pdf_tex}}

%fonty landscape i td
\usepackage[T1]{fontenc} % Use 8-bit encoding that has 256 glyphs
\usepackage[utf8]{inputenc} % Required for including letters with accents
\usepackage{anyfontsize} 
\usepackage{lscape}

%matma
\usepackage{enumitem} % Required for manipulating the whitespace between and within lists
\usepackage{amsmath,amssymb,amsthm} % For including math equations, theorems, symbols, etc
\usepackage{array}

%wykresy
\usepackage{pgfplots}
\usepackage{pgfplotstable}
\pgfplotsset{compat=newest}
\usepackage{booktabs}

%lepsze opisy
\usepackage{caption}
\usepackage{subcaption}

%kolory tabel
\usepackage{colortbl}


%----------------------------------------------------------------------------------------
%	THEOREM STYLES
%---------------------------------------------------------------------------------------

\theoremstyle{definition} % Define theorem styles here based on the definition style (used for definitions and examples)
\newtheorem{definition}{Definition}

\theoremstyle{plain} % Define theorem styles here based on the plain style (used for theorems, lemmas, propositions)
\newtheorem{theorem}{Theorem}

\theoremstyle{remark} % Define theorem styles here based on the remark style (used for remarks and notes)
