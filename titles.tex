\newcommand{\titleterma}[5]{
    % 1.numer sprawka
    % 2.temat
    % 3.nr grupy (N00-x0)
    % 4.data wykonania(dzien.miesiac)
    % 5.data oddania
    \begin{center}
        \fontsize{18}{1}\selectfont{ WYDZIAŁ MECHANICZNO-ENERGETYCZNY\\\vspace{10pt}
        POLITECHNIKI WROCŁAWSKIEJ }\\ \vspace{10pt}
        \textbf{\fontsize{26}{30}\selectfont LABORATORIUM TERMODYNAMIKI\\
        INSTYTUTU TECHNIKI CIEPLNEJ I MECHANIKI PŁYNÓW\\}
        \fontsize{16}{30}\selectfont Sprawozdanie nr. #1\\
        \textbf{Temat:}#2\\
        
    \end{center}
    \begin{flushleft}
        Grupa nr.#3 \\
        Skład podgrupy:
        \begin{enumerate}
            \item Grzegorz Wyborski 260906
            \item Anna Ziobrowska 255583
            \item Kacper Kasprzak 
        \end{enumerate}
        Termin zajęć: Czwartek, 18:55 20:30 \\
        Prowadzący: Mgr inż. Daria Krasota\\
        Data wykonania ćwiczenia: #4.2022 r.\\ 
        Data oddania sprawozdania: #5.2022 r.\\
        \vspace{20pt}
        \underline{Sprawozdanie powinno zawierać:}\\
        \begin{enumerate}
            \itemsep2pt{
            \item Podstawy teoretyczne
            \item Schemat układu pomiarowego
            \item Wykaz przyrządów pomiarowych
            \item Tabele pomiarowo-wynikowe
            \item Przykłady obliczeń
            \item Wykresy podanych zależności
            \item Uwagi, spostrzeżenia i wnioski
            \item Podpisany protokół z badań}
        \end{enumerate}
    \end{flushleft}
    \pagebreak
}

\newcommand{\titleelektra}[5]{
    \begin{center}
        \fontsize{18}{1}\selectfont{ Wydział Mechaniczno-Energetyczny\\
            Politechnika Wrocławska }\\ \vspace{10pt}
        \textbf{\fontsize{26}{30}\selectfont Podstawy Elektroniki\\ LABORATORIUM\\}
        \fontsize{16}{30}\selectfont Sprawozdanie nr. #1\\
        \textbf{Temat:}#2\\
        
    \end{center}
    \begin{flushleft}
        Grupa nr.#3 \\
        Skład podgrupy:
        \begin{enumerate}
            \item Grzegorz Wyborski 260906
            \item Ernest Kauch 260878
            \item Wojciech Mazur 252222
        \end{enumerate}
        Termin zajęć: Czwartek, 11:15-13.00 \\
        Prowadzący: Dr inż. Zbigniew Rogala\\
        Data wykonania ćwiczenia: #4.2022 r.\\ 
        Data oddania sprawozdania: #5.2022 r.\\
        \vspace{20pt}
        \underline{Sprawozdanie powinno zawierać:}\\
        \begin{enumerate}
            \itemsep 2pt{
            \item Podstawy teoretyczne
            \item Schemat układu pomiarowego
            \item Wykaz przyrządów pomiarowych
            \item Tabele pomiarowo-wynikowe
            \item Przykłady obliczeń
            \item Wykresy podanych zależności
            \item Uwagi, spostrzeżenia i wnioski
            \item Podpisany protokół z badań}
        \end{enumerate}
    \end{flushleft}
    \pagebreak
}

\newcommand{\titlebiomasa}[5]{
    \begin{center}
        \fontsize{18}{1}\selectfont{ Wydział Mechaniczno-Energetyczny\\
            Politechnika Wrocławska }\\ \vspace{10pt}
        \textbf{\fontsize{26}{30}\selectfont BIOMASA W ENERGETYCE\\ LABORATORIUM\\}
        \fontsize{16}{30}\selectfont Sprawozdanie nr. #1\\
        \textbf{Temat:}#2\\
    \end{center}
    \begin{flushleft}
        Grupa nr.#3 \\
        Skład podgrupy:
        \begin{enumerate}
            \item Grzegorz Wyborski 260906
            \item Szymon Murdza
        \end{enumerate}
        Termin zajęć: Czwartek, 15:15--16:55 \\
        Prowadzący: Dr inż. Krzysztof Mościcki\\
        Data wykonania ćwiczenia: #4.2022 r.\\ 
        Data oddania sprawozdania: #5.2022 r.\\
        \vspace{20pt}
        \underline{Sprawozdanie powinno zawierać:}\\
        \begin{enumerate}[\itemsep=2pt]
            \item Podstawy teoretyczne
            \item  Schemat układu pomiarowego
            \item Wykaz przyrządów pomiarowych
            \item Tabele pomiarowo-wynikowe
            \item Przykłady obliczeń
            \item Wykresy podanych zależności
            \item Uwagi, spostrzeżenia i wnioski
            \item Podpisany protokół z badań
        \end{enumerate}
    \end{flushleft}
    \pagebreak
}