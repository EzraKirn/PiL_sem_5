\subsection{Opis wymiennika}

\paragraph{}{
    Projektowany będzie wymiennika typu ``Rura w rurze - rury gięte``, jest to modyfikacja wymiennika typu ``Rura w rurze - rury proste``. 
    Stosowany jest w sytuacjach kiedy nie mamy wystarczającej ilości miejsca na zbudowanie klasycznego wymiwnnika z rurami prostymi. Dzięki swoim stosunkowo małym rozmiarą wymiennik ten znalazł zastosowanie w urządzeniach chłodniczych takicha jak lodówki. 
    Przestrzeń wewnątrz wymiennika moża być z łatwoącią zagospodarowana przez umieszczenie wewnątrz zbiornika naczynnik, sprężarki lub innego urządzenia.
    Jego największą wadą jest to, że w trakcie procesu zginania wewętrzna rura może wejś kontakt z rurą zewnętrzną od strony gięcia.
    Taka sytuacja  tworzy przestrzeń w której nie zachodzi wymiana ciepłą między czynnikami.
    Problem ten rozwiązuje się umieszczają pomiędzy rurami różne rodzaje dystansów.
}