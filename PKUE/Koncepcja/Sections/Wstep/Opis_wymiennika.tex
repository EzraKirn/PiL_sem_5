\subsection{Opis wymiennika}

\paragraph{}{
    Projektowany będzie wymiennik typu ''Rura w rurze-rury zwijane'', jest to modyfikacja wymiennika typu ''Rura w rurze-rury proste''.
    Stosowany jest w sytuacjach, kiedy nie mamy wystarczającej ilości miejsca na zbudowanie klasycznego wymiennika z rurami prostymi. Dzięki swoim stosunkowo małym rozmiarom wymiennik ten znalazł zastosowanie w urządzeniach chłodniczych takich jak lodówki.
    Przestrzeń wewnątrz wymiennika może być z łatwością zagospodarowana przez umieszczenie wewnątrz zbiornika na czynnik, sprężarki lub innego urządzenia.
    Jego największą wadą jest to, że w trakcie procesu zginania wewnętrzna rura może wejść kontakt z rurą zewnętrzną od strony gięcia.
    Taka sytuacja tworzy przestrzeń, w której nie zachodzi wymiana ciepła między czynnikami.
    Problem ten rozwiązuje się poprzez umieszczenie pomiędzy rurami różnego rodzaju dystansów.
}