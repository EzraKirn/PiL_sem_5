\subsection{Wybór materiałów oraz technologii}
\paragraph{}{
    Jako główny materiał wymiennika planuję wykorzystać aluminiowe rury gładkie.
    Rury zewnętrzna i wewnętrzna zostaną rozdzielone poprzez nawinięcie spirali z drutu aluminiowego.
    Z uwagi na łatwopalność toluenu króćce przyłączeniowe powinny zostać wykonane z mosiądzu, gdyż jest metalem nieiskrzącym.
    Nie ma potrzeby produkowania wewnętrznej rury ze stali nierdzewnej, ponieważ toluen nie reaguje z aluminium lub miedzią.
}
\paragraph{}{
    Elementy wymiennika zostaną połączone przy pomocy lutu miękkiego cynowo-ołowiowego z antymonem S-Pb58Sn40Sb2 (wg normy DIN EN 29453).
    Końce rury zewnętrznej zostaną uformowane na zimno tak, aby dokładnie przylegały do rury wewnętrznej.
    Wymiennik zostanie uformowany w spiralę na planie koła w procesie gięcia na zimno.
}
\paragraph{}{
    W wypadku kiedy nie zostaną osiągnięte odpowiednie temperatury na wylocie czynnika chłodzonego, zastosuję rury żebrowane zamiast spirali z drutu, w celu zwiększenia powierzchni wymiany ciepła.
    Ze wzgląd na bardzo wysokie koszty chciałbym uniknąć wykorzystani miedzi w wymienniku.
}