\subsection{Wybór materiałów oraz technoligii}
\paragraph{}{
    Bazowo planuję wykorzystać aluminiowę rury gładkie jako główny materiał wymiennika.
    Rury zewnetrzna i wewnętrzna zostaną rozdzielone poprzez nawinięcie spirali z drutu aluminiowego.
    Z uwagi na łatwopalność toulenu króćce przylączeniowe powinny zostać wykonane z mosiądzu, gdyż jest metalem nieiskrzącym.
    Nie ma potrzeby produkowania wewnętrznej rury ze stali nierdzewnej ponieważ toulen nie reagule z aluminiumub miedzią.
}
\paragraph{}{
    Elementy wymie zostaną połączone przy pomocy lutu miękkiego cynowo-ołowiowego z antymonem S-Pb58Sn40Sb2 (wg. normy DIN EN 29453).
    Końce rury zewnetrznej zostaną uformowane na zimno tak aby dokłądnie przylegały do rury wewnętrznej.
    Wymiennik zostanie uformowany w spirlę na planie koła w procesia gięcia na zimno.
}
\paragraph{}{
    W wypadku kiedy nie zostaną osiągnięte odpowiednie temperatury na wylocie czynnika chłodzonego zastosuję rury żebrowane zamiast spirali, w celu zwiększenia powierzchni wymiany ciepła.
    Ze wzglądy na bardz wysokie koszty chciałbym unuknąć wykorzystani miedzi w wymienniku.
}