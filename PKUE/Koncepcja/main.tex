\documentclass[
12pt, % Main document font size
a4paper, % Paper type, use 'letterpaper' for US Letter paper
oneside, % One page layout (no page indentation)
%twoside, % Two page layout (page indentation for binding and different headers)
]{article}

\include{../../structure.tex}
\newcommand{\titleterma}[5]{
    % 1.numer sprawka
    % 2.temat
    % 3.nr grupy (N00-x0)
    % 4.data wykonania(dzien.miesiac)
    % 5.data oddania
    \begin{center}
        \fontsize{18}{1}\selectfont{ WYDZIAŁ MECHANICZNO-ENERGETYCZNY\\\vspace{10pt}
        POLITECHNIKI WROCŁAWSKIEJ }\\ \vspace{10pt}
        \textbf{\fontsize{26}{30}\selectfont LABORATORIUM TERMODYNAMIKI\\
        INSTYTUTU TECHNIKI CIEPLNEJ I MECHANIKI PŁYNÓW\\}
        \fontsize{16}{30}\selectfont Sprawozdanie nr. #1\\
        \textbf{Temat:}#2\\
        
    \end{center}
    \begin{flushleft}
        Grupa nr.#3 \\
        Skład podgrupy:
        \begin{enumerate}
            \item Grzegorz Wyborski 260906
            \item Anna Ziobrowska 255583
            \item Kacper Kasprzak 
        \end{enumerate}
        Termin zajęć: Czwartek, 18:55 20:30 \\
        Prowadzący: Mgr inż. Daria Krasota\\
        Data wykonania ćwiczenia: #4.2022 r.\\ 
        Data oddania sprawozdania: #5.2022 r.\\
        \vspace{20pt}
        \underline{Sprawozdanie powinno zawierać:}\\
        \begin{enumerate}
            \itemsep2pt{
            \item Podstawy teoretyczne
            \item Schemat układu pomiarowego
            \item Wykaz przyrządów pomiarowych
            \item Tabele pomiarowo-wynikowe
            \item Przykłady obliczeń
            \item Wykresy podanych zależności
            \item Uwagi, spostrzeżenia i wnioski
            \item Podpisany protokół z badań}
        \end{enumerate}
    \end{flushleft}
    \pagebreak
}

\newcommand{\titleelektra}[5]{
    \begin{center}
        \fontsize{18}{1}\selectfont{ Wydział Mechaniczno-Energetyczny\\
            Politechnika Wrocławska }\\ \vspace{10pt}
        \textbf{\fontsize{26}{30}\selectfont Podstawy Elektroniki\\ LABORATORIUM\\}
        \fontsize{16}{30}\selectfont Sprawozdanie nr. #1\\
        \textbf{Temat:}#2\\
        
    \end{center}
    \begin{flushleft}
        Grupa nr.#3 \\
        Skład podgrupy:
        \begin{enumerate}
            \item Grzegorz Wyborski 260906
            \item Ernest Kauch 260878
            \item Wojciech Mazur 252222
        \end{enumerate}
        Termin zajęć: Czwartek, 11:15-13.00 \\
        Prowadzący: Dr inż. Zbigniew Rogala\\
        Data wykonania ćwiczenia: #4.2022 r.\\ 
        Data oddania sprawozdania: #5.2022 r.\\
        \vspace{20pt}
        \underline{Sprawozdanie powinno zawierać:}\\
        \begin{enumerate}
            \itemsep 2pt{
            \item Podstawy teoretyczne
            \item Schemat układu pomiarowego
            \item Wykaz przyrządów pomiarowych
            \item Tabele pomiarowo-wynikowe
            \item Przykłady obliczeń
            \item Wykresy podanych zależności
            \item Uwagi, spostrzeżenia i wnioski
            \item Podpisany protokół z badań}
        \end{enumerate}
    \end{flushleft}
    \pagebreak
}

\newcommand{\titlebiomasa}[5]{
    \begin{center}
        \fontsize{18}{1}\selectfont{ Wydział Mechaniczno-Energetyczny\\
            Politechnika Wrocławska }\\ \vspace{10pt}
        \textbf{\fontsize{26}{30}\selectfont BIOMASA W ENERGETYCE\\ LABORATORIUM\\}
        \fontsize{16}{30}\selectfont Sprawozdanie nr. #1\\
        \textbf{Temat:}#2\\
    \end{center}
    \begin{flushleft}
        Grupa nr.#3 \\
        Skład podgrupy:
        \begin{enumerate}
            \item Grzegorz Wyborski 260906
            \item Szymon Murdza
        \end{enumerate}
        Termin zajęć: Czwartek, 15:15--16:55 \\
        Prowadzący: Dr inż. Krzysztof Mościcki\\
        Data wykonania ćwiczenia: #4.2022 r.\\ 
        Data oddania sprawozdania: #5.2022 r.\\
        \vspace{20pt}
        \underline{Sprawozdanie powinno zawierać:}\\
        \begin{enumerate}[\itemsep=2pt]
            \item Podstawy teoretyczne
            \item  Schemat układu pomiarowego
            \item Wykaz przyrządów pomiarowych
            \item Tabele pomiarowo-wynikowe
            \item Przykłady obliczeń
            \item Wykresy podanych zależności
            \item Uwagi, spostrzeżenia i wnioski
            \item Podpisany protokół z badań
        \end{enumerate}
    \end{flushleft}
    \pagebreak
}
\include{../../units.tex}

\begin{document}
    \titlePUKE{18}{temat}{N02-20g}{17.03}{24.03}

    %\includepdf{Figures/PDF/Protokol_pomiarowy.pdf}

    \tableofcontents
    % \listoffigures
    % \listoftables
    % \begin{landscape}
    %     \begin{figure}
    %         \section{Stanowisko pomiarowe}
    %         \centering
    %         \includegraphics[width=\linewidth-142px]
    %             {Figures/PDF/Stanowisko_pomiarowe.pdf}
    %             \caption{Shemat budowy stanowiska pomiarowego z wyszczególnionymi elementami oraz pokazanym sposobem podlączenia.}
    %     \end{figure}
    % \end{landscape}
    \section{Wstęp}
    \subsection{Opis wymiennika}

\paragraph{}{
    Projektowany będzie wymiennik typu ''Rura w rurze-rury zwijane'', jest to modyfikacja wymiennika typu ''Rura w rurze-rury proste''.
    Stosowany jest w sytuacjach, kiedy nie mamy wystarczającej ilości miejsca na zbudowanie klasycznego wymiennika z rurami prostymi. Dzięki swoim stosunkowo małym rozmiarom wymiennik ten znalazł zastosowanie w urządzeniach chłodniczych takich jak lodówki.
    Przestrzeń wewnątrz wymiennika może być z łatwością zagospodarowana przez umieszczenie wewnątrz zbiornika na czynnik, sprężarki lub innego urządzenia.
    Jego największą wadą jest to, że w trakcie procesu zginania wewnętrzna rura może wejść kontakt z rurą zewnętrzną od strony gięcia.
    Taka sytuacja tworzy przestrzeń, w której nie zachodzi wymiana ciepła między czynnikami.
    Problem ten rozwiązuje się poprzez umieszczenie pomiędzy rurami różnego rodzaju dystansów.
}
    \subsection{Wybór materiałów oraz technoligii}
\paragraph{}{
    Bazowo planuję wykorzystać aluminiowę rury gładkie jako główny materiał wymiennika.
    Rury zewnetrzna i wewnętrzna zostaną rozdzielone poprzez nawinięcie spirali z drutu aluminiowego.
    Z uwagi na łatwopalność toulenu króćce przylączeniowe powinny zostać wykonane z mosiądzu, gdyż jest metalem nieiskrzącym.
    Nie ma potrzeby produkowania wewnętrznej rury ze stali nierdzewnej ponieważ toulen nie reagule z aluminiumub miedzią.
}
\paragraph{}{
    W wypadku kiedy nie zostaną osiągnięte odpowiednie temperatury na wylocie czynnika chłodzonego zastosuję rury żebrowane zamiast spirali, w celu zwiększenia powierzchni wymiany ciepła.
    Ze wzglądy na bardz wysokie koszty chciałbym unuknąć wykorzystani miedzi w wymienniku.
}
    \pagebreak
\subsection{Rysunki}
    \vspace{20px}
    \begin{figure}[ht]
        \centering
        \includegraphics[width=\linewidth-250px]{Przeciwprąd.png}
            \caption{Uproszczony schemat wymiennika ''Rura w rurze'' w układzie przepływu przeciwprądowego oraz wykres przedstawiający temperaturę czynników dla takiego układu.}
        \centering
        \includegraphics[width=\linewidth-250px]{Wymiennik.PNG}
            \caption{Poglądowy rysunek wymiennika typu ''Rura w rurze-rury gięte''.}
    \end{figure}



    \section{Charakterystyka techniczna}
    \subsection{Dane wejściowe}

    \begin{enumerate}
        \item 
            \begin{flushleft}
                Parametry cieczy schładzanej
                \begin{itemize}
                    \item Toluen
                    \item Temperatura wejściowa \(T_{1we} = 75\DEGc\)
                    \item Temperatura wyjściowa \(T_{1wy} = 55\DEGc\)
                    \item Strumień masy    \(m_{1} = 3\FLOWRATEkgs\)
                    \item Dodatkowe parametry czynnika dla \(T=T_{1średnie}=\frac{T_{1we}+T_{1wy}}{2}=65\DEGc\)
                        \begin{figure}[h]
                            \centering
                            \includegraphics[width=\linewidth-130px]{Tabela_toluen.PNG}
                        \end{figure}
                \end{itemize}
            \end{flushleft} 
        \pagebreak
        \item 
            \begin{flushleft}
                Parametry cieczy chłodzącej
                \begin{itemize}
                    \item Woda
                    \item Temperatura wejściowa \(T_{2we} = 10\DEGc\)
                    \item Temperatura wyjściowa \(T_{2wy} = 30\DEGc\)
                    \item Prędkość przepływu    \(U_{1} = 0.1-1\VELOCITY\)
                    \item Dodatkowe parametry czynnika \item Dodatkowe parametry czynnika dla \(T=T_{2średnie}=\frac{T_{2we}+T_{2wy}}{2}=20\DEGc\)
                        \begin{figure}[h]
                            \centering
                            \includegraphics[width=\linewidth-130px]{Tabela_woda.PNG}
                        \end{figure}
                    \end{itemize}
            \end{flushleft}
        \pagebreak    
    \end{enumerate}
    


    \subsection{Dokładne parametry materiałów}
\begin{itemize}
    \item Główny materiał wymiennika: Aluminium AW-1050 \\
    Własności wytrzymałościowe materiału wg normy DIN EN 755-2\\
    Parametry termodynamiczne materiału wg artykułu\href{https://www.azom.com/article.aspx?ArticleID=2798}{Amuminium 1050}
    
    \item Materiał króćców: Mosiądz CW501L\\
    Własności wytrzymałościowe materiału wg normy DIN EN 12163
\end{itemize}    
    \include{Sections/Wykres_przepywu.tex}

    \include{Sections/Podsumowanie.tex}
     
    %\bibliographystyle{plain} 
    %\bibliography{main}
\end{document}